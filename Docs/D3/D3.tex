% !TeX root = ./D3.tex
\documentclass[]{article}

% Imported Packages
%------------------------------------------------------------------------------
\usepackage{amssymb}
\usepackage{amstext}
\usepackage{amsthm}
\usepackage{amsmath}
\usepackage{enumerate}
\usepackage{fancyhdr}
\usepackage[margin=1in]{geometry}
\usepackage{graphicx}
\usepackage{extarrows}
\usepackage{setspace}
\usepackage{float}
\usepackage{multirow}
%------------------------------------------------------------------------------

% Header and Footer
%------------------------------------------------------------------------------
\pagestyle{plain}  
\renewcommand\headrulewidth{0.4pt}                                      
\renewcommand\footrulewidth{0.4pt}                                    
%------------------------------------------------------------------------------

% Title Details
%------------------------------------------------------------------------------
\title{
    \textbf{Group 10 - Deliverable \#3}\\
    \large SFWRENG 3A04: Software Design III - Large System Design\\
    \large Tutorial 1
}
\author{
    Andrew Hum 400138826\\
    Arkin Modi 400142497\\
    Hongzhao Tan 400136957\\
    Christopher Vishnu 400129743\\
    Shengchen Zhou 400050783\\
}
\date{March 20, 2020}           
%------------------------------------------------------------------------------

% Document
%------------------------------------------------------------------------------
\begin{document}

\maketitle
\newpage

\section{Introduction}
\label{sec:introduction}


\subsection{Purpose}
\label{sub:purpose}
The purpose of this document is to demonstrate HackerSim through various 
Unified Modelling Language (UML) diagrams. Utilizing the use-case diagrams and 
business events from Deliverable 1, as well as the analysis class diagram, 
architectural diagram and class responsibility collaboration (CRC) cards from 
Deliverable 2, we can create effective and detailed state diagrams, sequence 
diagrams and a class diagram. These UML diagrams will outline the states, 
sequence of interactions and core data structures that HackerSim is compiled 
of. This document is intended for the project manager, current members of the 
project team, future developers and stakeholders for the HackerSim project.

\subsection{System Description}
\label{sub:system_description}
The HackerSim system is an interactive game that will allow the user to raise a
Software Engineer (SE) in their room. The user will have various control 
options over their SE to assist them in their growth as well as the ability to 
customize their Software Engineer to their liking. These control options 
include choosing when the SE will work, play games, exercise, sleep and eat. 
Working will increase the SE’s currency and provide them with the ability to 
purchase various items for their room or food for when they’re hungry. Each of 
these activities will also directly affect the SE’s attributes in some manner 
and may extend or reduce its lifespan. Users will also have the ability to add 
friends and converse with them.\\

\noindent HackerSim utilizes a Presentation-Abstraction-Control (PAC) 
architecture that partitions the system into agents (triads) each containing a 
controller, entity object, and various presentation views. The system is broken
down into the following agents: General (Room), User Attributes, SE Attributes,
Shop, SE Inventory, Projects, Time-Step, and Friends \& Chat. These agents 
communicate with one another to control the data flow through the entire system
that is HackerSim.


\subsection{Overview}
\label{sub:overview}
The document is organized by the following sections: State Charts for 
Controller Classes, Sequence Diagrams and Detailed Class Diagram.

\begin{itemize}
    \item[] Section 2 contains State Chart Diagrams, built from the controllers 
    from the Analysis Class Diagram from Deliverable 2, that depicts the 
    system’s states and the events required for transitions

    \item[] Section 3 - contains Sequence Diagrams, built from the Use Case 
    Diagrams and Business Events from Deliverable 1, that visually explain the 
    flow of interactions between the user and the system

    \item[] Section 4 contains a Class Diagram, built from the CRC cards from 
    Deliverable 2, that defines the core data structures of the system
\end{itemize}

\section{State Charts for Controller Classes}
\label{sec:state_charts_for_controller_classes}
\begin{figure}[H]
    \centering
    \includegraphics[width=\textwidth]{"State Charts/State Chart - SE's Attributes Controller".png}
    \caption{State Chart - SE’s Attributes}
\end{figure}

\begin{figure}[H]
    \centering
    \includegraphics[width=\textwidth]{"State Charts/State Chart - Time-Step Controller".png}
    \caption{State Chart - Time-Step Controller}
\end{figure}

\begin{figure}[H]
    \centering
    \includegraphics[width=\textwidth]{"State Charts/State Chart - Project".png}
    \caption{State Chart - Projects}
\end{figure}

\begin{figure}[H]
    \centering
    \includegraphics[width=\textwidth]{"State Charts/State Chart - Friends and Chat".png}
    \caption{State Chart - Friends and Chat}
\end{figure}

\begin{figure}[H]
    \centering
    \includegraphics[width=\textwidth]{"State Charts/State Chart - User Attributes".png}
    \caption{State Chart - User Attributes}
\end{figure}

\begin{figure}[H]
    \centering
    \includegraphics[width=\textwidth]{"State Charts/State Chart - General Room".png}
    \caption{State Chart - General Room}
\end{figure}

\begin{figure}[H]
    \centering
    \includegraphics[width=\textwidth]{"State Charts/State Chart - Shop".png}
    \caption{State Chart - Shop}
\end{figure}

\begin{figure}[H]
    \centering
    \includegraphics[width=\textwidth]{"State Charts/State Chart - Inventory".png}
    \caption{State Chart - Inventory}
\end{figure}

\section{Sequence Diagrams}
\label{sec:sequence_diagrams}
\begin{figure}[H]
    \centering
    \includegraphics[width=\textwidth]{"Sequence Diagrams/Sequence Diagram - BE1".png}
    \caption{Sequence Diagram - User creates SE}
\end{figure}

\begin{figure}[H]
    \centering
    \includegraphics[width=\textwidth]{"Sequence Diagrams/Sequence Diagram - BE2".png}
    \caption{Sequence Diagram - SE Purchases Items and Skills}
\end{figure}

\begin{figure}[H]
    \centering
    \includegraphics[width=\textwidth]{"Sequence Diagrams/Sequence Diagram - BE3".png}
    \caption{Sequence Diagram - SE Works on Projects}
\end{figure}

\begin{figure}[H]
    \centering
    \includegraphics[width=\textwidth]{"Sequence Diagrams/Sequence Diagram - BE4".png}
    \caption{Sequence Diagram - Discrete-Time Pass}
\end{figure}

\begin{figure}[H]
    \centering
    \includegraphics[width=\textwidth, trim={0 7cm 0 0}, clip]{"Sequence Diagrams/Sequence Diagram - BE5".png}
    \caption{Sequence Diagram - User Interaction with Room Objects}
\end{figure}

\begin{figure}[H]
    \centering
    \includegraphics[width=\textwidth]{"Sequence Diagrams/Sequence Diagram - BE6".png}
    \caption{Sequence Diagram - Access Inventory and Use Items}
\end{figure}

\begin{figure}[H]
    \centering
    \includegraphics[width=\textwidth]{"Sequence Diagrams/Sequence Diagram - BE7".png}
    \caption{Sequence Diagram - Communication with Friends via Chat}
\end{figure}

\section{Detailed Class Diagram}
\label{sec:detailed_class_diagram}
\begin{figure}[H]
    \centering
    \makebox[\textwidth][c]{\includegraphics[width=1.2\textwidth]{"Detailed Class Diagram".png}}
    \caption{Detailed Class Diagram - HackerSim}
\end{figure}

\newpage
\appendix
\section{Division of Labour}
\label{sec:division_of_labour}
\begin{table}[H]
    \centering
    \caption{Division of Labour}
    \begin{tabular}{|p{2.5cm}|p{4cm}|p{9cm}|}
        \hline
        \textbf{Section} & \textbf{Contributor(s)} & \textbf{Description}\\
        \hline
        \textbf{Introduction} & Andrew Hum & Completed section and revision\\
        \hline
        \multirow{4}{2.5cm}{\textbf{State Charts}}  & Arkin Modi,           & Arkin: Projects, Friends \& Chat\\
        ~                                           & Christopher Vishnu,   & Chris: User Attributes, General (Room)\\
        ~                                           & Hongzhao Tan,         & Hongzhao: SE Attributes, Time-Step\\
        ~                                           & Shengchen Zhou        & Shengchen: Shop, SE Inventory\\
        \hline
        \multirow{5}{2.5cm}{\textbf{Sequence Diagrams}} & Andrew Hum,           & Andrew: Figure 11 \& 13\\
        ~                                               & Arkin Modi,           & Arkin: Figure 10 \& 15\\
        ~                                               & Christopher Vishnu,   & Chris: Figure 14\\
        ~                                               & Hongzhao Tan,         & Hongzhao: Figure 12\\
        ~                                               & Shengchen Zhao        & Shengchen: Figure 9\\
        \hline
        \textbf{Detailed Class Diagram} & Andrew Hum & Completed section and revision\\
        \hline
    \end{tabular}
\end{table}

\vspace{2cm}

\begin{table}[H]
    \begin{tabular}{p{5cm}}
    \\
    \hline
    Andrew Hum
    \\\\\\\\
    \hline
    Arkin Modi
    \\\\\\\\
    \hline
    Hongzhao Tan
    \\\\\\\\
    \hline
    Christopher Vishnu
    \\\\\\\\
    \hline
    Shengchen Zhou
    \end{tabular}
\end{table}

\end{document}
%------------------------------------------------------------------------------